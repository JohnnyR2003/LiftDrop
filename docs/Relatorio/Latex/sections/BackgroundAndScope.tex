\section{Background and Project Scope}
\subsection{Background}

The rise of the gig economy has led to the widespread adoption of mobile platforms for urban logistics, enabling flexible, on-demand delivery through services like Uber Eats, Glovo, and Deliveo. These applications rely on real-time geolocation, cloud-based infrastructure, and efficient routing to connect couriers and clients in fast-paced environments.

In developing \textbf{LiftDrop}, we sought to build on these foundational ideas by prioritizing clarity, responsiveness, and user autonomy. Our goal is to design a system that supports couriers through timely communication, real-time updates, and context-aware feedback.

\vspace{2mm}

\subsection*{Technical Inspirations}

LiftDrop incorporates several key technologies and design principles:

\begin{itemize}
    \item Real-time geolocation and distance-based filtering via the Google Maps API
    \item Bidirectional communication between couriers and the server using WebSockets
    \item A modular backend architecture that supports role-based access for clients and couriers.
\end{itemize}

\vspace{2mm}

\subsection{Project Scope}

This section defines the functional and technical boundaries of the LiftDrop platform. It outlines the key features available to clients and couriers, the strategies for real-time data communication and order assignment. It also clarifies what the system is designed to accomplish and what lies outside the current scope of the project.

\subsection{Platform Features}

\subsubsection{Client Features}
\begin{itemize}
    \item Register with email, password, and address;
    \item Login with email and password;
    \item Add a drop-off point when placing an order;
    \item Place orders by selecting restaurants, items, and a drop-off point;
\end{itemize}

\subsubsection{Courier Features}
\begin{itemize}
    \item Register as a courier with name, email, and password;
    \item Login as a courier with email and password;
    \item Accept or decline delivery requests;
    \item Set availability by entering or exiting "listening" status;
    \item Cancel an accepted order;
    \item Confirm pickup with a code;
    \item Confirm delivery with a code upon completion;
\end{itemize}

\subsection{Real-Time Communication Strategy}

Real-time data communication is crucial for dynamic interactions and courier management. Three communication models were considered:

\begin{itemize}
    \item \textbf{Polling (Request-Response):} Simple but inefficient, causing excessive network overhead.
    \item \textbf{Server-Sent Events (SSE):} Effective for unidirectional updates, but limited in interaction complexity.
    \item \textbf{WebSockets:} Full-duplex communication that minimizes latency, ideal for bidirectional updates like order assignments, courier status changes, and live notifications.
\end{itemize}

\textbf{WebSockets} were chosen for their ability to support real-time, bidirectional communication, ensuring dynamic and efficient interaction between couriers and the system.

\subsection{Order Assignment Strategy}

Order assignments consider the following criteria to ensure efficiency:

\begin{itemize}
    \item \textbf{Proximity:} Couriers closest to the pickup location are prioritized.
    \item \textbf{Traffic Conditions:} Real-time traffic data helps optimize delivery time.
    \item \textbf{Courier Rating: } Courier Ratings are taken in consideration when choosing the order for which couriers receive orders.
\end{itemize}

The system integrates the \textbf{Google Maps API} for geospatial data, route optimization, and traffic-aware services to support efficient order assignment.

\subsection{User Roles}
The platform operates with distinct user roles:
\begin{itemize}
    \item \textbf{Clients:} Individuals who place orders for delivery. They can track their orders, add drop-off points, and receive updates on delivery status.
    \item \textbf{Couriers:} Individuals who fulfill delivery orders. They can accept or reject assignments, set availability, and confirm pickups and deliveries.
\end{itemize}
