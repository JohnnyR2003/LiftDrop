\section{Scope}

\subsection{Scope Features}  

\subsubsection{Client Features}  
\begin{itemize}
    \item Place an order by selecting a restaurant and desired items.
    \item Track the order status and receive estimated time of arrival.
\end{itemize}

\subsubsection{Courier Features}  
\begin{itemize}
    \item Accept or decline incoming orders.
    \item Set availability by starting or stopping waiting status.
    \item Update order status after acceptance.
    \item Cancel an order if needed.
    \item Confirm delivery upon completion.
\end{itemize}

\subsubsection{User Features}  
\begin{itemize}
    \item Register as either a client or a courier.
    \item Clients provide personal and contact information during registration.
    \item Couriers specify their means of transportation upon registration.
\end{itemize}
  

\subsection{Real-Time Data Handling and Order Assignment}  

Developing a mobile application focused on delivery presents challenges, particularly in real-time data handling and order assignment. Since multiple couriers interact with the platform simultaneously, the system must manage updates effectively while ensuring a smooth user experience. After research and consideration, we identified three primary approaches to handling real-time updates and concurrency:

\begin{itemize}
    \item \textbf{Polling-based Request-Response Model} – The client periodically requests updates from the server. While simple to implement, this approach results in unnecessary network requests, increased server load, and delayed updates.
    \item \textbf{WebSockets} – A bidirectional communication protocol that allows both the client and server to send messages dynamically. While WebSockets are ideal for interactive, two-way communication (e.g., chat applications), SSE offers a simpler and more reliable approach for our needs, focusing on server-driven updates.
    \item \textbf{Event-Driven Server-Sent Events (SSE)} – The server pushes updates to clients as they occur, enabling real-time communication without the inefficiencies of constant polling. This is well-suited for unidirectional data flow (server-to-client updates).
\end{itemize}

Given the requirements of this project, we chose \textbf{SSE} as the preferred method for real-time updates due to its simplicity, maintainability, and efficient handling of one-way event streams. To implement SSE on Android, we will use OkHttp since Android lacks native SSE support.

\subsection{Order Assignment Strategy}  

For order assignment, the platform will consider factors such as:

\begin{itemize}
    \item \textbf{Proximity-based allocation} – Assigning orders based on the courier’s distance from the pickup location.
    \item \textbf{Fair distribution} – Ensuring a balanced order assignment among available couriers.
    \item \textbf{Traffic-aware routing} – Taking real-time traffic conditions into account when assigning deliveries.
\end{itemize}

An external geospatial API will handle location-based calculations (distance, routing, and traffic), simplifying the implementation by reducing the need for complex in-house geospatial logic. We selected the \textbf{Google Maps API} as it offers the most suitable free plan for the required functionalities.

\subsection{Safety Measures}  

To enhance safety, the platform will feature \textbf{neighborhood safety ratings} based on feedback from couriers who have completed deliveries in those areas. Only couriers can submit ratings, ensuring assessments reflect real delivery conditions.  

Additionally, couriers will receive alerts for high-risk areas, and \textbf{nighttime safety measures} will highlight risk-prone locations for late-night deliveries.