\section{Limitations and Future Work}

While LiftDrop delivers a functional prototype focused on real-time delivery coordination and transparent courier interaction, several ideas and features were considered but not implemented due to time constraints or project scope boundaries.

\subsection{Unimplemented Features}

\begin{itemize}
    \item \textbf{Client Application Interface}: The current system includes a simulated client API for testing, but no dedicated client-facing mobile application was developed.
    
    \item \textbf{Administrative Dashboard}: A management panel for monitoring orders and courier activity.
    
    \item \textbf{Push Notifications}: All real-time updates currently rely on WebSockets; platform-native push notifications could enhance user engagement and reliability.

    \item \textbf{WebSocket Reconnection Support}: The WebSocket implementation currently lacks fault-tolerance mechanisms for automatic reconnection in case of network disruptions. This may affect delivery request communication stability on unreliable connections.
    
    \item \textbf{Courier Wait Time Compensation}: The current earnings model does not factor in courier wait time at pickup locations. Including this would ensure fairer compensation for delays caused by clients or merchants.
\end{itemize}

\subsection{Future Improvements}

\begin{itemize}
    \item \textbf{Internationalization (i18n)}: Support for multiple languages to make the app more inclusive for international users.

    \item \textbf{Resilient Real-Time Communication}: Replace or enhance WebSocket infrastructure with support for automatic reconnection and state synchronization to improve robustness in unstable network environments.

    \item \textbf{Enhanced Earnings Model}: Extend the current earnings calculation to include compensation for time spent waiting at pickup locations, in addition to distance and item value.

    \item \textbf{Courier Performance Analytics}: Introduce metrics for tracking delivery efficiency, response times, and reliability.
    
    \item \textbf{Scalability Testing}: The platform could benefit from stress testing under simulated high-traffic scenarios to identify performance bottlenecks.
\end{itemize}

This list reflects the balance between development priorities and available resources and serves as a guide for future iterations of the LiftDrop platform.
