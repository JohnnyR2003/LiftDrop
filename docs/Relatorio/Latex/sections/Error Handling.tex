\subsection{Error Handling}

To ensure robustness and clarity in the face of runtime failures, LiftDrop adopts a layered error-handling strategy. This architecture supports clear diagnostics, consistent logging, user-oriented feedback, and modular server behavior—principles influenced by best practices studied in the DAW course.

\subsubsection{Result Modeling with \texttt{Either}}

Internally, success and failure outcomes are represented using a functional-style result type:

\begin{lstlisting}[language=Kotlin, caption={Functional Result Modeling with Either}]
sealed class Either<out L, out R> {
    data class Left<out L>(val value: L) : Either<L, Nothing>()
    data class Right<out R>(val value: R) : Either<Nothing, R>()
}

fun <R> success(value: R) = Either.Right(value)
fun <L> failure(error: L) = Either.Left(error)
\end{lstlisting}

This model ensures that failure handling is explicit and composable, improving testability and reducing reliance on exceptions.

\newpage

\subsubsection{Standardized API Errors with Problem Details}

For all HTTP interactions, the backend follows the \textbf{RFC 7807 Problem Details for HTTP APIs} specification. This standard defines a JSON structure for expressing errors in a machine-readable way:

\begin{lstlisting}[language=Kotlin, caption={Problem Details Model}]
data class Problem(
    val type: String?,
    val title: String?,
    val status: Int?,
    val detail: String?,
    val instance: String? = null
)
\end{lstlisting}

Responses use the \texttt{application/problem+json} media type and include fields such as \texttt{status}, \texttt{title}, and \texttt{detail} to clearly describe what went wrong.

A centralized factory exposes named constructors like:
\begin{itemize}
  \item \texttt{courierNotNearPickup()}
  \item \texttt{ratingAlreadyDone()}
  \item \texttt{userAlreadyExists()}
  \item \texttt{invalidRequestContent()}
\end{itemize}

Each problem type links to documentation hosted at:

\texttt{https://github.com/isel-sw-projects/2025-lift-drop/tree/main/docs/problems}

\subsubsection{Global Exception Handling}

All errors not handled by business logic are intercepted by a custom Spring \texttt{@ControllerAdvice}, shown below:

\begin{lstlisting}[language=Kotlin, caption={Global Exception Handler}]
@ControllerAdvice
class CustomExceptionHandler : ResponseEntityExceptionHandler() {

    override fun handleMethodArgumentNotValid(...) =
        Problem.invalidRequestContent("Argument received is not valid")
               .response(HttpStatus.BAD_REQUEST)

    override fun handleTypeMismatch(...) =
        Problem.invalidRequestContent("There is a type mismatch")
               .response(HttpStatus.BAD_REQUEST)


    
    override fun handleHttpMessageNotReadable(...) =
        Problem.invalidRequestContent("Http message is not readable")
               .response(HttpStatus.BAD_REQUEST)

    @ExceptionHandler(Exception::class)
    fun handleAllExceptions(ex: Exception, request: WebRequest): ResponseEntity<String> {
        GlobalLogger.log("Http.CustomExceptionHandler AllExceptions: $ex")
        return ResponseEntity.status(HttpStatus.INTERNAL_SERVER_ERROR)
                             .body("Internal Server Error")
    }
}
\end{lstlisting}

This handler ensures:
\begin{itemize}
    \item Graceful fallback for unexpected issues.
    \item Logged diagnostics.
    \item Consistent error responses for clients.
\end{itemize}

\subsubsection{Frontend Feedback Integration}

Errors returned in \texttt{Problem} format are parsed and used in the Android application to display appropriate UI feedback (e.g., invalid login, wrong pickup code, courier not near drop-off).
