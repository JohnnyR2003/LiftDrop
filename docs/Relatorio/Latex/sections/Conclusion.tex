\section{Conclusion}

The development of the LiftDrop platform offered a practical exploration of the challenges involved in building real-time, location-aware delivery systems. Through the design and implementation of a courier-focused mobile application and supporting backend infrastructure, we aimed to simulate core components of modern gig economy services in a simplified but functional form.

Key aspects such as real-time communication via WebSockets, distance-based order assignment, and session-based authentication were successfully integrated and tested. Additionally, the modular architecture and transaction-safe database access layers provide a strong foundation for future enhancements and scalability.

Beyond technical implementation, this project allowed us to engage with broader questions around user experience, fairness, and system transparency. Although we chose not to implement features such as fairness algorithms or complex routing logic, the decisions we made were informed by both practical constraints and real-world platform behavior.

Ultimately, LiftDrop stands as a focused, extensible prototype that reflects our ability to apply software engineering principles, explore unfamiliar technologies, and reason through architectural trade-offs. Future work could expand the platform’s capabilities with client-facing applications, analytics, and more sophisticated operational tools.
